\documentclass[12pt,a4paper]{article}

% --- Packages ---
\usepackage[spanish]{babel}
\usepackage[utf8]{inputenc}
\usepackage[T1]{fontenc}
\usepackage{geometry}
\usepackage{graphicx}
\usepackage{float}
\usepackage{booktabs}
\usepackage{tabularx}
\usepackage{longtable}
\usepackage{enumitem}
\usepackage{hyperref}
\usepackage{tikz}
\usetikzlibrary{arrows.meta, positioning, shapes, calc}
\usepackage{caption}
\usepackage{subcaption}
\usepackage{xcolor}
\usepackage{fontawesome5}
\usepackage{fancyhdr}
\usepackage{titlesec}

\geometry{
  left=25mm,
  right=25mm,
  top=25mm,
  bottom=25mm
}

\hypersetup{
  colorlinks=true,
  linkcolor=teal,
  urlcolor=blue,
  citecolor=teal,
  pdfauthor={Equipo Ready4Hire},
  pdftitle={Ready4Hire - Informe Técnico del Asistente Conversacional}
}

\renewcommand{\familydefault}{\sfdefault}

\pagestyle{fancy}
\fancyhf{}
\fancyhead[L]{Ready4Hire}
\fancyhead[R]{Informe Técnico del Asistente}
\fancyfoot[C]{\thepage}

\titleformat{\section}{\large\bfseries}{\thesection}{1em}{}
\titleformat{\subsection}{\normalsize\bfseries}{\thesubsection}{1em}{}
\titleformat{\subsubsection}{\normalsize\itshape}{\thesubsubsection}{1em}{}

\setlist[itemize]{nosep,left=1.5em}
\setlist[enumerate]{nosep,left=1.5em,label=\arabic*.}

% --- Custom colors ---
\definecolor{r4hTeal}{HTML}{007F7F}
\definecolor{r4hGray}{HTML}{444444}

\begin{document}

\begin{titlepage}
  \centering
  {\Large \textbf{Ready4Hire – Informe Técnico del Asistente Conversacional}\\[1ex]}
  \rule{\textwidth}{0.6pt}\\[2ex]
  {\large{\textbf{Curso:} Ingeniería de Software – 2025-2}}\\[1ex]
  {\large{\textbf{Fecha:} 11 de noviembre de 2025}}\\[1ex]
  {\large{\textbf{Equipo:}} Elizabeth Suescún (SM), Luis M. Marín (PO), Santiago Manco (Líder Técnico),\\
  Juan E. Villada (Frontend), Jerónimo Restrepo (IA), Emanuel Torres (Backend), Santiago Arboleda (QA)}\\[1ex]
  {\large{\textbf{Repositorio:}} \url{https://github.com/JeronimoRestrepo48/Ready4Hire}}\\[4ex]
  \includegraphics[width=0.5\textwidth]{figures/placeholder_portada.png}\\[2ex]
  {\small \textit{Nota: Sustituir la imagen anterior por la pieza gráfica oficial del proyecto.}}\\[4ex]
  \vfill
  {\large \textbf{Resumen}}\\[1ex]
  \begin{minipage}{0.9\textwidth}
    \small
    El presente informe técnico documenta el diseño conceptual del asistente conversacional Ready4Hire, 
    incluyendo su propósito, público objetivo, flujos de interacción, arquitectura tecnológica y principales herramientas 
    utilizadas. El documento incorpora espacios destinados a evidencias y capturas de pruebas que respaldan la funcionalidad 
    alcanzada durante la fase final del curso.
  \end{minipage}
  \vfill
\end{titlepage}

\tableofcontents
\listoffigures
\newpage

\section{Descripción general del asistente}
\subsection{Propósito}
Ready4Hire es un mentor virtual que simula entrevistas técnicas y comportamentales para aspirantes a cargos de tecnología. 
El asistente orienta al candidato en tres etapas:
\begin{enumerate}
  \item \textbf{Preparación previa:} contextualiza al usuario según rol y nivel, y entrega preguntas diagnósticas.
  \item \textbf{Ejecución guiada:} mantiene una conversación multimodal (texto/voz) respaldada por modelos de lenguaje grandes (LLM).
  \item \textbf{Retroalimentación accionable:} ofrece puntajes automáticos, análisis emocional y rutas de mejora personalizadas.
\end{enumerate}

La combinación de evaluación inteligente y diseño centrado en el usuario permite cerrar la brecha entre teoría y práctica para 
estudiantes y profesionales junior.

\subsection{Público objetivo}
\begin{itemize}
  \item Estudiantes de ingeniería de sistemas que buscan practicar entrevistas técnicas reales.
  \item Graduados recientes y bootcampers que requieren feedback inmediato y estructurado.
  \item Centros de talento universitarios interesados en generar reportes y métricas de desempeño.
  \item Reclutadores aliados que desean medir de forma objetiva competencias técnicas y blandas.
\end{itemize}

\section{Ejemplos de interacción}
\subsection{Escenario 1 – Selección de entrevista}
\begin{quote}\small
\textbf{Usuario:} Quiero practicar para un rol de Backend mid-level.\\
\textbf{Asistente:} Perfecto. Adaptaré las preguntas a microservicios, resiliencia y CI/CD. ¿Listo para iniciar?\\
\textbf{Usuario:} Sí.\\
\textbf{Asistente:} Primera pregunta: Describe cómo diseñarías un servicio de autenticación con tokens rotativos.
\end{quote}

\subsection{Escenario 2 – Retroalimentación y coaching}
\begin{quote}\small
\textbf{Usuario:} Esta es mi respuesta...\\
\textbf{Asistente:} Destacaste segregación de responsabilidades, pero faltó mencionar mecanismos de \\ rate limiting. 
Te sugiero revisar patrones Circuit Breaker y registrar métricas de auditoría.\\
\textbf{Asistente:} Puntuación global 7.8/10. Emoción detectada: neutral. Próxima pregunta: ¿cómo automatizarías este despliegue?
\end{quote}

\section{Arquitectura del asistente}
\subsection{Visión general}
La solución integra un frontend Blazor Server, un backend FastAPI diseñado con Domain-Driven Design (DDD), servicios de IA 
especializados y una capa de datos híbrida (PostgreSQL, Redis, Qdrant). La Figura~\ref{fig:arquitectura} resume los componentes.

\begin{figure}[H]
  \centering
  \begin{tikzpicture}[
    node distance=1.5cm,
    every node/.style={font=\small},
    box/.style={draw=r4hTeal, rounded corners, thick, align=center, minimum width=3.6cm, minimum height=1.05cm, fill=r4hTeal!6},
    service/.style={draw=r4hGray, rounded corners, thick, align=center, minimum width=3.1cm, minimum height=1.05cm, fill=white},
    datastore/.style={draw=black, thick, cylinder, shape border rotate=90, aspect=0.22, minimum height=1.4cm, minimum width=1.0cm, fill=gray!10},
    arrow/.style={-{Latex[length=3mm]}, thick}
  ]
    % Línea principal
    \node[box] (user) {\faUser\ Usuario final\\Canales Web / móvil};
    \node[box, below=of user] (blazor) {WebApp Blazor Server\\\textit{UI + MVVM + SignalR}};
    \node[box, below=of blazor] (api) {API FastAPI\\\textit{DDD + CQRS + Validaciones}};
    \node[box, below=of api] (core) {Aplicación de Dominio\\Casos de uso, Agregados};
    \node[box, below=of core] (integration) {Integración y Observabilidad\\Circuit Breaker, Telemetría};

    % Servicios laterales (más cerca)
    \node[service, left=2.6cm of api] (services) {Servicios IA\\\small LLM (Ollama), Whisper STT, Recomendador};
    \node[service, right=2.6cm of api] (workers) {Procesos asíncronos\\\small Celery + Redis broker};
    \node[service, right=2.6cm of core] (security) {Capa de seguridad\\\small JWT, auditoría, sanitización};

    % Datos inferior
    \node[datastore, below left=0.1cm and 1.8cm of integration] (postgres) {PostgreSQL\\Persistencia relacional};
    \node[datastore, below=0.1cm of integration] (redis) {Redis\\Caché y rate limiting};
    \node[datastore, below right=0.1cm and 1.8cm of integration] (qdrant) {Qdrant\\Embeddings de preguntas};

    % Flechas verticales
    \draw[arrow] (user) -- node[right]{UI responsiva, SSO} (blazor);
    \draw[arrow] (blazor) -- node[right]{SignalR / HTTPS} (api);
    \draw[arrow] (api) -- node[right]{DTOs Pydantic} (core);
    \draw[arrow] (core) -- node[right]{Eventos de dominio} (integration);
    \draw[arrow] (integration) -- (redis);

    % Flechas laterales
    \draw[arrow] (api.west) -- node[above]{REST / gRPC} (services.east);
    \draw[arrow] (api.east) -- node[above]{Queues} (workers.west);
    \draw[arrow] (core.east) -- node[above]{Policies, IAM} (security.west);
    \draw[arrow, dashed] (services.south east) |- (integration.north);
    \draw[arrow, dashed] (workers.south west) |- (integration.north);
    \draw[arrow, dashed] (security.south) |- (integration.north);

    % Datos
    \draw[arrow] (integration) -- (postgres);
    \draw[arrow] (integration) -- (qdrant);
    \draw[arrow, dashed] (services.south) |- (qdrant.north);
  \end{tikzpicture}
  \caption{Arquitectura lógica del asistente Ready4Hire (flujo principal vertical dentro del margen).}
  \label{fig:arquitectura}
\end{figure}

\subsection{Componentes técnicos destacados}
\subsubsection*{WebApp Blazor Server}
El frontend opera bajo el patrón MVVM utilizando componentes Razor y SignalR para sincronización en tiempo real. Se orquesta el estado global con contenedores de dependencia propios, soporta modo PWA (service worker + cache storage) y delega autenticación al backend vía JWT.

\subsubsection*{API FastAPI y capa de dominio}
El backend adopta Clean Architecture: routers ligeros, servicios de aplicación (casos de uso) y entidades de dominio rigen las reglas de negocio. Los DTOs Pydantic incluyen validaciones estrictas, control de versiones y esquemas OpenAPI generados automáticamente.

\subsubsection*{Servicios de IA y pipelines de evaluación}
Los prompts se gestionan a través de LangChain, con Ollama como motor local para respuestas evaluativas y resumen de feedback. Whisper procesa audio multilenguaje en modo batch; un motor de recomendación interno ajusta la dificultad usando embeddings (Sentence Transformers) almacenados en Qdrant.

\subsubsection*{Procesamiento asíncrono y resiliencia}
Celery coordina evaluaciones pesadas y generación de certificados; Redis actúa como broker y como caché de resultados. Se implementan patrones Circuit Breaker y Retry con tenacidad configurable para llamadas a LLM.

\subsubsection*{Observabilidad y cumplimiento}
OpenTelemetry exporta trazas hacia Prometheus y Grafana; los logs estructurados siguen formato JSON y se enriquecen con IDs de conversación. Los módulos de seguridad aplican hardening: sanitización de entrada, guardas de prompt injection, auditoría de decisiones y políticas de rate limiting contextual.

\subsection{Flujo conversacional detallado}
\begin{figure}[H]
  \centering
  \begin{tikzpicture}[
    node distance=1.5cm and 2cm,
    process/.style={rectangle, rounded corners, draw=r4hGray, thick, minimum width=3.4cm, minimum height=1cm, align=center, fill=white},
    decision/.style={diamond, aspect=2, draw=r4hGray, thick, minimum width=2.8cm, align=center, fill=white},
    arrow/.style={-{Latex[length=3mm]}, thick},
    font=\small
  ]
    \node[process] (start) {Inicio sesión usuario};
    \node[process, below=of start] (select) {Selección de rol y dificultad};
    \node[process, below=of select] (context) {Preguntas de contexto};
    \node[process, below=of context] (llm) {Generación pregunta con LLM};
    \node[process, below=of llm] (answer) {Respuesta del candidato (texto/voz)};
    \node[process, below=of answer] (evaluation) {Evaluación automática (Score + emoción)};
    \node[decision, below=of evaluation] (continue) {¿Quedan preguntas?};
    \node[process, below left=1.8cm and 2cm of continue] (feedback) {Entrega feedback y consejos};
    \node[process, below right=1.8cm and 2cm of continue] (report) {Generación informe y badges};
    \node[process, below=of feedback] (nextq) {Siguiente pregunta adaptativa};
    \node[process, below=of report] (end) {Cierre de sesión y exportación};

    \draw[arrow] (start) -- (select);
    \draw[arrow] (select) -- (context);
    \draw[arrow] (context) -- (llm);
    \draw[arrow] (llm) -- (answer);
    \draw[arrow] (answer) -- (evaluation);
    \draw[arrow] (evaluation) -- (continue);
    \draw[arrow] (continue) -- node[left]{Sí} (feedback);
    \draw[arrow] (feedback) -- (nextq);
    \draw[arrow] (nextq) |- (llm);
    \draw[arrow] (continue) -- node[right]{No} (report);
    \draw[arrow] (report) -- (end);
  \end{tikzpicture}
  \caption{Flujo conversacional de punta a punta.}
  \label{fig:flujo}
\end{figure}

\subsection{Tecnologías y herramientas}
\begin{table}[H]
  \centering
  \caption{Resumen de tecnologías clave del asistente.}
  \label{tab:tecnologias}
  \begin{tabularx}{\textwidth}{@{}lXl@{}}
    \toprule
    \textbf{Capa} & \textbf{Tecnología / Servicio} & \textbf{Propósito principal} \\
    \midrule
    Frontend & Blazor Server (.NET 9), Chart.js, Service Worker PWA & UI, dashboards y modo offline \\
    Backend & FastAPI (Python 3.11), Clean Architecture, SOLID & Casos de uso, control de dominio \\
    IA & Ollama (Llama 3.2), LangChain, Whisper & Evaluación semántica, STT \\
    Datos & PostgreSQL, Redis, Qdrant & Persistencia relacional, caché, vector search \\
    Observabilidad & Prometheus, Grafana, OpenTelemetry & Monitoreo, métricas, trazas \\
    Automatización & Celery, Docker Compose, Playwright & Tareas asíncronas, despliegue local, pruebas E2E \\
    \bottomrule
  \end{tabularx}
\end{table}

\section{Evidencias de pruebas y capturas}
\subsection{Pruebas funcionales}
\begin{figure}[H]
  \centering
  \fbox{\rule{0pt}{6cm}\rule{0.9\textwidth}{0pt}}
  \caption{Captura de la simulación en Blazor mostrando feedback en tiempo real.}
  \label{fig:captura_blazor}
\end{figure}

\subsection{Pruebas automatizadas}
\begin{figure}[H]
  \centering
  \fbox{\rule{0pt}{6cm}\rule{0.9\textwidth}{0pt}}
  \caption{Resultados de pruebas Playwright / pytest (insertar captura).}
  \label{fig:captura_tests}
\end{figure}

\subsection{Dashboards de observabilidad}
\begin{figure}[H]
  \centering
  \fbox{\rule{0pt}{6cm}\rule{0.9\textwidth}{0pt}}
  \caption{Dashboard Grafana con métricas de latencia y score promedio.}
  \label{fig:captura_grafana}
\end{figure}

\section{Referencias}
\begin{itemize}
  \item LangChain. (2024). \textit{LangChain Documentation}. Recuperado de \url{https://www.langchain.com/}
  \item Hugging Face. (2024). \textit{Transformers Library}. Recuperado de \url{https://huggingface.co/docs/transformers}
  \item OpenAI. (2024). \textit{Whisper Speech-to-Text}. Recuperado de \url{https://github.com/openai/whisper}
  \item Microsoft. (2024). \textit{Blazor Documentation}. Recuperado de \url{https://learn.microsoft.com/aspnet/core/blazor/}
  \item Qdrant. (2025). \textit{Vector Database Documentation}. Recuperado de \url{https://qdrant.tech/documentation/}
\end{itemize}

\section*{Anexos}
\addcontentsline{toc}{section}{Anexos}
\subsection*{Anexo A – Checklist de despliegue}
\begin{longtable}{@{}p{4.5cm}p{9cm}@{}}
  \toprule
  \textbf{Paso} & \textbf{Detalle técnico} \\ \midrule
  Preparar entorno & Crear archivo \texttt{.env} con claves (JWT, Redis, Qdrant, API Keys). Verificar versiones: Python 3.11+, Node 20+, .NET 9. \\ \midrule
  Servicios base & Ejecutar \texttt{docker-compose up -d redis qdrant prometheus grafana}. Revisar salud con \texttt{docker ps}. \\ \midrule
  Backend API & Activar entorno virtual: \texttt{source venv/bin/activate}. Correr migraciones (si aplica) y lanzar \texttt{uvicorn app.main\_v2\_improved:app --host 0.0.0.0 --port 8001}. \\ \midrule
  WebApp Blazor & Desde \texttt{WebApp/}, compilar con \texttt{dotnet build}. Lanzar \texttt{dotnet run --urls=https://localhost:5214}. \\ \midrule
  Observabilidad & Cargar dashboards en Grafana (importar JSON). Confirmar trazas en \texttt{http://localhost:3000}. \\ \midrule
  Smoke tests & Ejecutar \texttt{pytest tests/smoke} y \texttt{npx playwright test --grep "smoke"}. Registrar resultados en la bitácora. \\ \bottomrule
\end{longtable}

\subsection*{Anexo B – Plantilla de registro de experimentos A/B}
\begin{longtable}{@{}p{4cm}p{9.5cm}@{}}
  \toprule
  \textbf{Campo} & \textbf{Descripción / ejemplo} \\ \midrule
  Nombre del experimento & E.g., "Selector de preguntas adaptativo v2". \\ \midrule
  Objetivo & Hipótesis y métrica primaria (p.ej., aumentar score promedio en 10"). \\ \midrule
  Métricas asociadas & Precisión LLM, completitud entrevistas, tiempo promedio de respuesta. \\ \midrule
  Variantes y tráfico & Variante A (baseline), Variante B (nuevo algoritmo). Distribución sugerida 50/50. \\ \midrule
  Setup técnico & Feature flag utilizado, ruta de despliegue, dependencias externas. \\ \midrule
  Resultados preliminares & Tabla con fechas, volumen de usuarios, métricas relevantes. \\ \midrule
  Interpretación & Conclusiones, sesgos detectados, decisiones de negocio. \\ \midrule
  Acción & Adoptar/descartar/iterar, responsable y fecha objetivo. \\ \bottomrule
\end{longtable}

\subsection*{Anexo C – Bitácora de incidentes y monitoreo}
\begin{longtable}{@{}p{2.5cm}p{3cm}p{7cm}@{}}
  \toprule
  \textbf{Fecha} & \textbf{Severidad} & \textbf{Descripción y resolución} \\ \midrule
  \rule{0pt}{2.6ex} & & (Registrar incidentes de infraestructura, interrupciones LLM, fallos de despliegue). \\ \bottomrule
\end{longtable}
\noindent\emph{Nota:} Actualizar este anexo semanalmente con datos reales extraídos de Prometheus/Grafana.

\subsection*{Anexo D – Evidencia de pruebas automatizadas}
\begin{itemize}
  \item Ruta de reporte Playwright: \texttt{e2e-tests/playwright-report/index.html}
  \item Ruta de cobertura pytest: \texttt{Ready4Hire/htmlcov/index.html}
  \item Checklist de ejecución CI/CD: \texttt{.github/workflows/tests.yml}
  \item Insertar capturas o métricas resumidas (p.ej., tabla con \\textit{commit}, fecha, resultado, cobertura).
\end{itemize}

\end{document}

